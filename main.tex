\documentclass[a4paper]{GreyWolfsScoutActivityTemplate}

\usepackage{GreyWolfsScoutFleurDeLis} % For drawing a Fleur-de-lis
\usepackage{lipsum}
%\usepackage {showframe}
\title{Scout Activity} % 
% Set your author text; this will be shown in the footnotes
\author{\href{mailto:mike.jones@mansouthscouts.org.uk}{Mike Jones} CSL @ \href{http://23rdMcr.uk}{23\textsuperscript{rd} Mcr Scout Grp}}

% Change the badge widths default 3cm
%\setlength{\badgewidth}{3cm}

% Change the licence in the footer. The default is:
%\licence{\href{https://creativecommons.org/licenses/by-sa/4.0/}{\color{black} \ccbysa}}

\begin{document}

\activitypart[
  SquirrelsActivityBadgeStoryTime={Now this is the story all about how. My life got flipped, turned upside down. And I'd like to take a minute just sit and wait,
  While I tell you how to use this \LaTeX{} template.},
  AllSectionsStagedActivityBadgeDigitalCitizen=There is no badge for Adults.,
  ExplorersActivityBadgeMediaRelations=I expect using this template is in the right ball park for the Media Relations badge. Unfortunately I've yet to add Explorers and Squirrels badges (except these two).]{\mbox{Activity documents}}

\Sections[scouts,cubs,explorers]

\section*{Introduction}
Some instructions on how to use this template to create a Scout branded Activity document.
This is a latex template based on the Scout Brand Centre's Activity Word Document template {\tt ms-word-document-template-activity.docx}. 

\begin{SummaryImageBox}{Scout-Activity-Electroetching-pp1}
\sessiontime[30 minutes]
\section*{Equipment}
\equip{Badge info}
\equip{Timings}
\equip{Section suitability}
\equip{Activity title}
\equip{Procedure}
\equip{This template!}
\subsection*{Optional}
\equip{Image}
\equip{Risk assessment}
\equip{Discussion ideas}
\end{SummaryImageBox}

\begin{ActivityProcedure}
\section{}
First you need this template! Have a look at the compiled PDF and the source {\tt main.tex}. 
In the preamble set your {\tt \textbackslash{}author\{\}} variable. 
\section{}
The rest of your contents will be placed between the {\tt \textbackslash{}begin\{document\}} and the {\tt \textbackslash{}end\{document\}}.
\section{}
To start each Activity \textit{part}, use the command {\tt \small{\textbackslash{}activitypart\{Title\}}}
\section{}
To list badges on the left-hand-side, you need to add an option to that command:\\ {\tt \small{\textbackslash{}activitypart\\\hspace*{5mm}[name=text,\ldots]\{\ldots\}}}. Where {\tt name} is the name of a badge as listed in section~\ref{badges} below, and {\tt text} is the text you wish to appear beneath it.
\section{}
Now use the {\tt \textbackslash{}Sections[]} command to specify which \textit{\color{ScoutPurple} Scout} sections this activity is suitable for, e.g.\\{\tt \small{\textbackslash{}Sections[scouts,cubs]}}
\section{}
If you wish you can add an introduction. Add the line {\tt \textbackslash{}section*\{Introduction\}} followed by text. You can use this method to add any section you like in the main body of the document. see e.g. the Risk Assessment section \ref{RA} below.

Use the asterisked versions if you wish ro avoid section numbering.
\section{}
Add a 2-column summary.

\subsection{}
To add a summary and graphic use the environment: {\tt SummaryImageBox} and specify a graphic you wish to use. e.g.\\
{\tt \textbackslash{}begin\{SummaryImageBox\}\\\hspace*{5mm}\{ImageLocation\}}\\
\%Insert left content here.\\
{\tt \textbackslash{}end\{SummaryImageBox\}}

\subsection{}
Alternatively, to add a summary-left and other-content-right use the environment: {\tt SummaryBox} e.g.\\ 
{\tt \textbackslash{}begin\{SummaryBox\}\\\hspace*{5mm}\{ImageLocation\}}\\
\%Insert left content here\\
{\tt \textbackslash{}SummaryBoxSep}\\
\%Insert right content here.\\
{\tt \textbackslash{}end\{SummaryBox\}}

\section{}
Suggestions for filling the \textit{summary} left-content:

\subsection{}
Specify Duration of session using this command: e.g.\\
{\tt \textbackslash{}sessiontime[30~minutes]}

\subsection{}
Add an equipment section:\\
{\tt \small{
\textbackslash{}section*\{Equipment\}\\
\textbackslash{}equip\{Badge info\}\\
\textbackslash{}equip\{Timings\}\\
\textbackslash{}equip\{Section suitability\}\\
\textbackslash{}equip\{Activity title\}\\
\textbackslash{}equip\{Procedure\}\\
\textbackslash{}equip\{This template!\}\\
\textbackslash{}subsection*\{Optional\}\\
\textbackslash{}equip\{Image\}\\
\textbackslash{}equip\{Risk assessment\}\\
\textbackslash{}equip\{Discussion ideas\}
}}

\subsection{}
End the Summary\ldots{}Box:\\
{\tt \textbackslash{}end\{SummaryImageBox\}}\\
or\\
{\tt \textbackslash{}end\{SummaryBox\}}

\section{}
To fill the step-by-step activity instructions now create an {\tt ActivityProcedure} environment. This will split up the flow of text into three columns and change the space and size of the heading.

\subsection{}
Use blank \textbackslash{}section\{\} and \textbackslash{}subsection\{\} commands to create numbered paragraphs.

\end{ActivityProcedure}



\section*{Discussion}
Use most any \LaTeX{} commands to fill remaining sections.
You could add:
\begin{itemize}
 \item a \textit{Discussion} section; 
 \item a \textit{Risk Assessment section};
 \item an \textit{Acknowledgement} section.
\end{itemize}



\section*{Risk Assessment}
\label{RA}
Have you done a DSE assessment? \href{https://www.hse.gov.uk/msd/dse/}{https://www.hse.gov.uk/msd/dse/}



\section*{Licence}
This Latex class, template and it's accompanying files collectively form the software known as Grey Wolf's Scouts Activity Template. Files from \href{https://github.com/nimpo/GreyWolfsScoutsBeamerTheme}{Grey Wolf's Scouts Beamer Theme} have been used here too.

\subsection*{Scout Branding}
The design follows the Scout Association's Branding template: \href{https://scoutsbrand.org.uk/catalogue/item/ms-word-document-template-activity}{ms-word-document-template-activity} and \href{https://docs.scoutsbrand.org.uk/guidelines.pdf}{guidelines} to best efforts.

\subsection*{Fonts}
The fonts chosen and supplied for use in this project are those specified by The Scout Association's branding guidelines: the Nunito Sans family has been chosen because it can be used at no cost and suits the chosen style.

Nunito Sans is distributed by Google as a TrueType Font and is freely
available under the terms of the SIL Open Font License licence
See:\\\href{https://fonts.google.com/specimen/Nunito+Sans\#license}{https://fonts.google.com/specimen/Nunito+Sans\#license}\\and\\\href{https://scripts.sil.org/cms/scripts/page.php?site_id=nrsi&id=OFL}{https://scripts.sil.org/cms/scripts/page.php?site\_id=nrsi\&id=OFL}

It was necessary to create a modified version of the Nunito Sans TTF to
enable them to work with this software.
\begin{itemize}
\item Downgrade the OS/2 table
\item Convert to Adobe Type 1 postscript in order to create the TFM files
\end{itemize}
The Derived TTF files are used in the output PDF.

\subsection*{Images}
Included in Grey Wolf's Scouts Activity Template's distribution are images of badges derived from those available on the \href{https://www.scouts.org.uk}{www.scouts.org.uk} and the various badge sponsors at the time of writing.

Also included are the Scout Logos. The permitted use of the various Scout Logos is defined in the \href{https://www.scouts.org.uk/por/14-other-matters/rule-147-protected-scout-logos-names-badges-and-awards/}{Scout POR: Rule 147}.

\subsection*{This PDF Document}
This compiled PDF document is licensed for distribution under the creative commons \href{https://creativecommons.org/licenses/by-sa/4.0/}{Attribution-ShareAlike 4.0 International (CC BY-SA 4.0) licence}.

\subsection*{Grey Wolf's Scouts Activity Template.}
Grey Wolf's Scouts Activity Template software is licensed under the terms of the GNU General Public License. I include a copy of the LICENCE/LICENSE verbatim below.  

\begin{verbatim}
%% Copyright 2022 Mike Jones, <dr.mike.jones@gmail.com>
%% AKA Grey Wolf <mike.jones@mansouthscouts.org>
%% AKA Akela <mike.jones@mansouthscouts.org>
%% [23rd Manchester (Birch with Fallowfield)]
%% Scout Membership number: 12114313
%
% This file is part of Grey Wolf's Scouts Activity Template.
%
% Grey Wolf's Scouts Activity Template is free software: you can redistribute
% it and/or modify it under the terms of the GNU General Public License 
% as published by the Free Software Foundation, either version 3 of the
% License, or (at your option) any later version.
%
% Grey Wolf's Scouts Activity Template is distributed in the hope that it will
% be useful, but WITHOUT ANY WARRANTY; without even the implied warranty
% of MERCHANTABILITY or FITNESS FOR A PARTICULAR PURPOSE.  See the GNU
% General Public License for more details.
%
% You should have received a copy of the GNU General Public License
% along with Grey Wolf's Scouts Activity Template.  If not, see 
% <https://www.gnu.org/licenses/>.
\end{verbatim}



\section*{Here's a SummaryBox example}
\begin{SummaryBox}
This is an example of the alternative to the {\tt \textbackslash{}SummaryImageBox} environment used on the first page.
\par \lipsum[1]
\SummaryBoxSep
\fleur[width=\textwidth,colour=ScoutPurple]
\par
\vspace{5mm}
\lipsum[2]
\end{SummaryBox}
\clearpage



\section{Badge Names}
\label{badges}
\subsection*{List of keys accepted by \textbackslash{}activitypart[key=text,\ldots]\{\}}
\vspace{-5mm}
\begin{multicols}{2}
\footnotesize
\badgelist
\end{multicols}
\end{document}
